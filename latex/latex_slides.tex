\documentclass[mathserif]{beamer}

\useoutertheme{split}
\useinnertheme{rectangles}
\usecolortheme{uiuc}
\logo{\includegraphics[height=0.5cm]{uiuclogo.pdf}}

\usepackage{beamerthemesplit} % TODO: use different beamer theme
\usepackage{framed}
%\usepackage{amsmath}

\title{SIAM: Getting Started with \LaTeX}
\author{Matthew Michelotti}
\date{\today}

\renewcommand{\arraystretch}{1.1}

\setbeamertemplate{navigation symbols}{}%remove navigation symbols
\setbeamertemplate{caption}[numbered]

\begin{document}

\defbeamertemplate*{footline}{shadow theme} {
\leavevmode \hbox{\begin{beamercolorbox}[wd=.5\paperwidth,
ht=2.5ex,dp=1.125ex,leftskip=.3cm plus1fill,rightskip=.3cm]
{author in head/foot}%
\usebeamerfont{author in head/foot}\insertshortauthor
\end{beamercolorbox}\begin{beamercolorbox}[wd=.5\paperwidth,
ht=2.5ex,dp=1.125ex,leftskip=.3cm,rightskip=.3cm plus1fil]
{title in head/foot}%
\usebeamerfont{title in head/foot}\insertshorttitle
\hfill \insertframenumber{} / \inserttotalframenumber
\end{beamercolorbox}}
}

\frame{\titlepage}

\begin{frame}[fragile]
  \frametitle{What is \LaTeX{}?}

  \begin{itemize}
  \item \LaTeX{} is a high-quality typesetting system
  \item \LaTeX{} markup is converted into nice looking pdf files
  \end{itemize}

  \begin{minipage}{.38\textwidth}
  \begin{framed}
    \tiny
    \begin{verbatim}\documentclass[12pt]{article}
\usepackage{amsmath}
\title{\LaTeX}
\date{}
\begin{document}
  \maketitle
  \section{Introduction}
  \LaTeX{} is a document
  preparation system for the
  \TeX{} typesetting program.
  It offers programmable desktop
  publishing features and
  ...
  \begin{align}
    E &= mc^2 \\
    m &= \frac{m_0}
      {\sqrt{1-\frac{v^2}{c^2}}}
  \end{align}
\end{document}\end{verbatim}
  \end{framed}
  \end{minipage}
  $\rightarrow$
  \begin{minipage}{.55\textwidth} 
  \begin{framed}
    \includegraphics[trim = 33mm 127mm 33mm 59mm, clip, width=\textwidth]{figures/example_article.pdf}
  \end{framed}
  \end{minipage}
\end{frame}

\begin{frame}[fragile]
  \frametitle{Why Use \LaTeX{}?}
  \begin{itemize}
    \item Produces high-quality documents
    \item Offers precise control over how document looks
    \item Excellent for typesetting mathematics
    \item Automated references, citations, etc.
    \item Widely used for academic journals
    \item Free
    \item Multi-platform
  \end{itemize}
\end{frame}

\begin{frame}[fragile]
  \frametitle{Compiling}
  \begin{itemize}
  \item Installing \LaTeX{} on your computer
  \begin{itemize}
    \item Mac: from MacPorts, MacTeX, TeXShop
    \item Windows: TeXworks, MiKTeX
    \item Helpful site: en.wikibooks.org/wiki/LaTeX/Installation
  \end{itemize}
  \item Use \verb|latexmk| or \verb|pdflatex| or \verb|latex| command on a .tex file to produce output
  \begin{itemize}
    \item \verb|latex| command is not recommended, since it cannot include .pdf, .jpg, .png image formats
      and cannot output to .pdf
  \end{itemize}
  \item Can also compile online at sharelatex.com
  \end{itemize}
  \begin{figure}
  \includegraphics[width=.2\textwidth]{figures/texshop.png} \quad
  \includegraphics[width=.22\textwidth]{figures/texworks.png} \quad
  \includegraphics[width=.2\textwidth]{figures/share_latex.jpg}
  \end{figure}
\end{frame}

%\frame{
%  \frametitle{Exercise 1: Compiling}
%  \begin{itemize}
%    \item If you have \LaTeX{} installed on your laptop, you can use that
%    \item Otherwise, go to www.compileonline.com/try\_latex\_online.php
%    \item Compile the example \LaTeX{} file on this website
%    \begin{itemize} \item The result should look as follows: \end{itemize}
%  \end{itemize}
%
%  \centerline{
%  \begin{minipage}{.6\textwidth} 
%  \begin{framed}
%    \includegraphics[trim = 33mm 127mm 33mm 59mm, clip, width=\textwidth]{figures/example_article.pdf}
%  \end{framed}
%  \end{minipage}
%  }
%}

\begin{frame}[fragile]
  \frametitle{Control Sequences}
  \begin{itemize}
    \item \LaTeX{} uses control sequences to achieve special functionality
    \item Control sequences start with a backslash \verb|\|
  \end{itemize}
  \small
  \begin{table}
  \begin{tabular}{l l}
    \verb|\documentclass[11pt]{article}| & describes appearance of document \\
     & (similar to CSS) \\
    \verb|\usepackage{amsmath}| & include package named amsmath \\
    \verb|\begin{document}| & begins document environment \\
    \verb|\section{Section Title}| & starts a new section \\
    \verb|\subsection{Subsection Title}| & starts a new subsection \\
    \verb|\LaTeX{}| & displays \LaTeX{} \\
    \verb|\end{document}| & ends document environment
  \end{tabular}
  \end{table}
\end{frame}

\begin{frame}[fragile]
  \frametitle{Example Document}
  \begin{itemize}
    \item \LaTeX{} document will have information like title and date in top matter
    \item Contents of document belong in document environment
  \end{itemize}
  \begin{verbatim}\documentclass[11pt]{article}
\usepackage{amsmath}
\title{\LaTeX}
\author{Your Name}
\date{} % omits date since this is empty
\begin{document}
  \maketitle
  \section{Introduction}
  This is the introduction of my document.
  ...
\end{document}\end{verbatim}
\end{frame}

\begin{frame}[fragile]
  \frametitle{Spaces}
{\small
\begin{verbatim}You can use LaTeX to typeset regular text.  In LaTeX, using
extra     spaces       or      a newline   doesn't matter.

However, using two newlines in a row results in a new
paragraph.

% line comments begin with a percent sign\end{verbatim}
}
  \begin{minipage}{\textwidth} 
  %\begin{framed}
    \includegraphics[trim = 33mm 230mm 33mm 40mm, clip, width=\textwidth]{figures/example_spacing.pdf}
  %\end{framed}
  \end{minipage}
\end{frame}

\begin{frame}[fragile]
  \frametitle{Cross-Referencing}
  \begin{itemize}
    \item Many things in \LaTeX{}, such as sections and subsections, are automatically numbered
    \begin{itemize}
      \item Numbering can be suppressed using asterisk, e.g.\ \verb|\section*{...}|
    \end{itemize}
    \item Automatically numbered entities in \LaTeX{} can be labeled using \verb|\label{...}|
    \item Any labeled entity can be referenced using \verb|\ref{...}|
    \item Command \verb|\ref{...}| is often preceded by \verb|~|, denoting a space that
      cannot be a line break
  \end{itemize}
  \begin{verbatim}\section{Introduction}
\label{sec:intro}
This is the introduction.
\section{Results}
In section~\ref{sec:intro}, we provided introductory
material. Now we will provide results.\end{verbatim}
\end{frame}

\begin{frame}[fragile]
  \frametitle{Compiling Multiple Times}
  \begin{itemize}
    \item When running \verb|pdflatex| on \verb|example.tex|, multiple files are created
    \begin{itemize}
      \item \verb|example.pdf|: output file
      \item \verb|example.aux|: contains auxiliary about references, etc.
      \item \verb|example.log|, \verb|example.synctex.gz|, etc.
    \end{itemize}
    \item Cross-reference information in .aux file is not ready until
      after \verb|pdflatex| is run
    \begin{itemize}
      \item May need to run \verb|pdflatex| twice for references to display correctly
    \end{itemize}
    \item Using the tool LaTeX-Mk can resolve this issue
    \begin{itemize}
      \item Use command \verb|latexmk -pdf <filename>.tex| to perform all operations needed to generate final pdf
    \end{itemize}
  \end{itemize}
\end{frame}

\frame{
  \frametitle{Exercise 1: Sections}
  \begin{itemize}
    \item Create the following document in \LaTeX{}
  \end{itemize}
  \centerline{
  \begin{minipage}{.6\textwidth} 
  \begin{framed}
    \includegraphics[trim = 33mm 115mm 33mm 59mm, clip, width=\textwidth]{figures/referencing.pdf}
  \end{framed}
  \end{minipage}
  }
  \begin{itemize}
    \item Helpful sites:
      \begin{itemize} \footnotesize
        \item http://en.wikibooks.org/wiki/LaTeX/Document\_Structure
        \item http://en.wikibooks.org/wiki/LaTeX/Labels\_and\_Cross-referencing
      \end{itemize}
  \end{itemize}
}

\begin{frame}[fragile]
  \frametitle{Exercise 1: Sections (Solution)}
{\small \begin{verbatim}\documentclass[12pt]{article}
\title{\LaTeX}
\author{Your Name}
\begin{document}
  \maketitle
  \section{Introduction}
  This is the introduction.  Section~\ref{sec:body} contains
  the body of the paper.
  \section{Body}
  \label{sec:body}
  This is the body.
    \subsection{Approach}
    This is a numbered subsection.
    \subsection*{Results}
    This is an unnumbered subsection.
\end{document}\end{verbatim} }
\end{frame}

\begin{frame}[fragile]
  \frametitle{Figures}
  \begin{itemize}
    \item Figures can be created using the figure environment
    \item Various options for placing figures
    \begin{itemize}
      \item \verb|h|: here, approximately
      \item \verb|t|: top
      \item \verb|b|: bottom
      \item \verb|p|: on its own page with other such figures
      \item For example, \verb|begin{figure}[ht]...| will place the figure
        approximately where it is listed in the markup and at the top of a page
    \end{itemize}
    \item Figures can include a caption and be labeled
  \end{itemize}
\end{frame}

\begin{frame}[fragile]
  \frametitle{Figures}
  \begin{itemize}
    \item The figure below was made with the following code:
  \end{itemize}
  \begin{verbatim}\begin{figure}
  \includegraphics[width=.6\textwidth]{siebel.jpg}
  \label{fig:siebel}
  \caption{A picture of Siebel Center}
\end{figure}\end{verbatim}
  \begin{figure}
     \includegraphics[width=.6\textwidth]{figures/siebel.jpg}
     \label{fig:siebel}
     \caption{A picture of Siebel Center}
  \end{figure}
\end{frame}

\begin{frame}[fragile]
  \frametitle{BibTeX}
  \begin{itemize}
    \item BibTeX is a tool used to cite articles/books and automatically form a bibliography
    \item Use \verb|cite| command to cite something in your paper
    \begin{itemize} \item Example: \verb|\cite{greenwade93}| \end{itemize}
    \item Use \verb|\bibliographystyle| and \verb|\bibliography| commands at the
      end of document where bibliography should be
    \begin{itemize}
      \item Example: \verb|\bibliographystyle{plain}|  \verb|\bibliography{references}{}|
    \end{itemize}
    \item Make a .bib file (called references.bib in our example) that describes each source
  \end{itemize}
\end{frame}

\begin{frame}[fragile]
  \frametitle{BibTeX}
  \begin{itemize}
    \item Example .bib file:
  \end{itemize}
{\footnotesize  \begin{verbatim}@article{greenwade93,
    author  = "George D. Greenwade",
    title   = "The {C}omprehensive {T}ex {A}rchive {N}etwork
              ({CTAN})",
    year    = "1993",
    journal = "TUGBoat",
    volume  = "14",
    number  = "3",
    pages   = "342--351"
}
@book{goossens93,
    author    = "Michel Goossens and Frank Mittelbach and
                 Alexander Samarin",
    title     = "The LaTeX Companion",
    year      = "1993",
    publisher = "Addison-Wesley",
    address   = "Reading, Massachusetts"
}\end{verbatim} }
\end{frame}

\begin{frame}[fragile]
  \frametitle{Math Mode}
  \begin{itemize}
    \item Text between dollar signs \verb|$...$| will use math mode
    \item Many control sequences only work in math mode
    \item Can use \verb|^| for superscripts and \verb|_| for subscripts
  \end{itemize}
  \begin{table}
  \begin{tabular}{l l}
    \verb|$y = 3x - 4$| & $\rightarrow \quad y = 3x - 4$ \\
    \verb|$\theta \Theta \omega \Omega$| & $\rightarrow \quad \theta \Theta \omega \Omega$\\
    \verb|$\sqrt{x} = x^{1/2}$| & $\rightarrow \quad \sqrt{x} = x^{1/2}$\\
    \verb|$\min \{ x_1, x_2, x_3 \}$| & $\rightarrow \quad \min \{ x_1, x_2, x_3 \}$
  \end{tabular}
  \end{table}
\end{frame}

\begin{frame}[fragile]
  \frametitle{Displayed Math}
  \begin{itemize}
    \item Example: {\small \verb|f(x) = \sum_{i=1}^{\infty} \frac{1}{g_i(x)}|}
    \item Use dollar signs \verb|$...$| for inline math
      \begin{itemize} \item The equation $f(x) = \sum_{i=1}^{\infty} \frac{1}{g_i(x)}$ is displayed inline \end{itemize}
    \item Use escaped brackets \verb|\[...\]| to display math on its own line
      \[ f(x) = \sum_{i=1}^{\infty} \frac{1}{g_i(x)} \]
    \item Use \verb|\begin{equation}...\end{equation}| for automatically numbered equations
      \begin{equation} f(x) = \sum_{i=1}^{\infty} \frac{1}{g_i(x)} \end{equation}
  \end{itemize}
\end{frame}

\begin{frame}[fragile]
  \frametitle{Exercise 2: Definition of Derivative}
  \begin{itemize}
    \item Produce the following equation in \LaTeX{}:
    \[ \frac{\mathrm{d}f(x)}{\mathrm{d}x} = \lim_{\delta \to 0} \frac{f(x + \delta) - f(x)}{\delta} \]
    \item Start from the following:
  \end{itemize}
\begin{verbatim}\documentclass[11pt]{article}
\usepackage{amsmath}
\begin{document}
  % add your content here
\end{document}\end{verbatim}
  \begin{itemize}
    \item Helpful sites:
      \begin{itemize}
        \item en.wikibooks.org/wiki/LaTeX/Mathematics
        \item ftp.ams.org/pub/tex/doc/amsmath/short-math-guide.pdf
      \end{itemize}
  \end{itemize}
\end{frame}

\begin{frame}[fragile]
  \frametitle{Exercise 2: Definition of Derivative (Solution)}
  \begin{itemize}
    \item Produce the following equation in \LaTeX{}:
    \[ \frac{\mathrm{d}f(x)}{\mathrm{d}x} = \lim_{\delta \to 0} \frac{f(x + \delta) - f(x)}{\delta} \]
    \item Solution:
  \end{itemize}
  {\small \begin{verbatim}\[
  \frac{\mathrm{d}f(x)}{\mathrm{d}x}
  = \lim_{\delta \to 0} \frac{f(x + \delta) - f(x)}{\delta}
\]\end{verbatim}}
  \begin{itemize}
    \item Do the $\mathrm{d}$s in your solution look different?
      \begin{itemize} \item \verb|\mathrm| displays upright characters in math mode \end{itemize}
  \end{itemize}
\end{frame}

\begin{frame}[fragile]
  \frametitle{Array-Like Environments}
  \begin{itemize}
    \item Use \verb|array| environment or one of the matrix environments to make table of information
    \item Matrix environments include delimiters for convenience
    \begin{itemize}
      \item \verb|pmatrix| $()$, \verb|bmatrix| $[]$, \verb|Bmatrix| $\{\}$,
      \verb|vmatrix| $| |$, \verb|Vmatrix| $\| \|$
    \end{itemize}
    \item Columns separated with \verb|&|, rows separated with \verb|\\|
%    \item Column alignment specified at start using \verb|l| for left, \verb|c| for center
%      and \verb|r| for right
%   \begin{itemize}\item Example: \verb|{ccc}| means use three centered columns\end{itemize}
  \end{itemize}

  \begin{minipage}{.47\textwidth}
  \begin{framed}
    \begin{verbatim}A = \begin{pmatrix}
      2  & -1 & 0  \\
      -1 & 2  & -1 \\
      0  & -1 & 2
    \end{pmatrix}\end{verbatim}
  \end{framed}
  \end{minipage}
  $\rightarrow$
  \begin{minipage}{.43\textwidth} 
  \begin{framed}
    \[
%      A = \left( \begin{array}{ccc}
%      2 & -1 & 0 \\
%      -1 & 2 & -1 \\
%      0 & -1 & 2 \end{array} \right)
A = \begin{pmatrix}
      2  & -1 & 0  \\
      -1 & 2  & -1 \\
      0  & -1 & 2
    \end{pmatrix}
    \]
  \end{framed}
  \end{minipage}
\end{frame}

\begin{frame}[fragile]
  \frametitle{Align Environment}
  \begin{itemize}
    \item Use \verb|align| environment to line up multiple equations
    \item Left and right sides of equation separated with \verb|&|
    \item Equations separated with \verb|\\|
    \item Using \verb|align*| instead of \verb|align| will suppress equation numbers
  \end{itemize}
  \begin{minipage}{.52\textwidth}
  \begin{framed}
    \begin{verbatim}\begin{align}
  x & = \cos(\theta(t)) \\
  y & = \sin(\theta(t)) \\
  \theta(t) &
     = \omega t + \phi
\end{align}\end{verbatim}
  \end{framed}
  \end{minipage}
  $\rightarrow$
  \begin{minipage}{.41\textwidth} 
  \begin{framed}
    \begin{align}
      x & = \cos(\theta(t)) \\
      y & = \sin(\theta(t)) \\
     \theta(t) & = \omega t + \phi
    \end{align}
  \end{framed}
  \end{minipage}
\end{frame}

\begin{frame}[fragile]
  \frametitle{Resizing Delimiters}
  \begin{itemize}
    \item The \verb|\left|, \verb|\right|, and \verb|\middle| commands are used to automatically resize
      delimiters like parenthesis based on content
    \item Period \verb|.| denotes an omitted left or right delimiter
    \item The \verb|\big|, \verb|\Big|, \verb|\bigg|, and \verb|\Bigg| commands can be used to manually resize delimiters
  \end{itemize}
  \begin{table}
  \small
  \begin{tabular}{l l}
    \verb|$\left(\frac{x^2}{y+x}\right)^2$| & $\rightarrow \quad$ \begin{minipage}{.3\textwidth} \[ \left(\frac{x^2}{y + x}\right)^2 \]\end{minipage} \\
    \begin{minipage}{.6\textwidth}\begin{verbatim}$P\left(X=1 \middle|
  \frac{X}{Y} \geq 2 \right)$\end{verbatim}\end{minipage} & $\rightarrow \quad$ \begin{minipage}{.3\textwidth} \[ P\left(X=1 \middle|\frac{X}{Y} \geq 2\right) \]\end{minipage} \\
    \verb!$\left.2x-\frac{2}{x^3}\right|_0^1$! & $\rightarrow \quad$ \begin{minipage}{.3\textwidth}\[ \left.2x - \frac{2}{x^3}\right|_0^1\] \end{minipage} \\
    \verb|$( \big( \Big( \bigg( \Bigg($| & $\rightarrow \quad$ \begin{minipage}{.3\textwidth}\[ ( \big( \Big( \bigg( \Bigg( \] \end{minipage} 
  \end{tabular}
  \end{table}
\end{frame}

\begin{frame}[fragile]
  \frametitle{Math Accents}
  \begin{itemize} \item Special accents over variables/expressions may be used in math mode \end{itemize}
  \begin{table}
  \begin{tabular}{l l l l}
    \verb|$\bar{x}$| & $\rightarrow \quad \bar{x}$ &
    \qquad \qquad \verb|$\vec{x}$| & $\rightarrow \quad \vec{x}$ \\
    \verb|$\dot{x}$| & $\rightarrow \quad \dot{x}$ &
    \qquad \qquad \verb|$\hat{x}$| & $\rightarrow \quad \hat{x}$ \\
    \verb|$\ddot{x}$| & $\rightarrow \quad \ddot{x}$ &
    \qquad \qquad \verb|$\tilde{x}$| & $\rightarrow \quad \tilde{x}$ \\
    \verb|$\acute{x}$| & $\rightarrow \quad \acute{x}$ &
    \qquad \qquad \verb|$\grave{x}$| & $\rightarrow \quad \grave{x}$
  \end{tabular}
  \end{table}
\end{frame}

\begin{frame}
  \frametitle{Exercise 3: More Math}
  \begin{itemize}
    \item Produce the following equations in \LaTeX{}:
      \begin{align*}
        \begin{vmatrix}a & b \\ c & d\end{vmatrix} & = ad - bc \\
        \hat{x_i} & = x_i \left(\sum_{j=0}^\infty x_j \right)^{-1}
      \end{align*}
    \item Helpful sites:
      \begin{itemize}
        \item en.wikibooks.org/wiki/LaTeX/Mathematics
        \item ftp.ams.org/pub/tex/doc/amsmath/short-math-guide.pdf
      \end{itemize}
  \end{itemize}
\end{frame}

\begin{frame}[fragile]
  \frametitle{Exercise 3: More Math (Solution)}
  \begin{itemize}
    \item Produce the following equations in \LaTeX{}:
      \begin{align*}
        \begin{vmatrix}a & b \\ c & d\end{vmatrix} & = ad - bc \\
        \hat{x_i} & = x_i \left(\sum_{j=0}^\infty x_j \right)^{-1}
      \end{align*}
    \item Solution:
  \end{itemize}
  {\small \begin{verbatim}\begin{align*}
  \begin{vmatrix} a & b \\ c & d \end{vmatrix} & = ad - bc \\
  \hat{x_i} & = x_i \left(\sum_{j=0}^\infty x_j \right)^{-1}
\end{align*}\end{verbatim}}
\end{frame}

\frame{
  \frametitle{If You Want to Know More \ldots}
  \begin{itemize}
    \item \emph{en.wikibooks.org/wiki/LaTeX/} : Excellent reference for \LaTeX{} in general
    \item \emph{www.ctan.org} : CTAN stands for the Comprehensive \TeX{} Archive Network,
      and contains many standard \LaTeX{} packages
  \end{itemize}
}

\end{document}

\end{document}
